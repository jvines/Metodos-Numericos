\documentclass[letter, 11pt]{article}
%% ================================
%% Packages =======================
\usepackage[utf8]{inputenc}      %%
\usepackage[T1]{fontenc}         %%
\usepackage{lmodern}             %%
\usepackage[spanish]{babel}      %%
\decimalpoint                    %%
\usepackage{fullpage}            %%
\usepackage{fancyhdr}            %%
\usepackage{graphicx}            %%
\usepackage{amsmath}             %%
\usepackage{color}               %%
\usepackage{mdframed}            %%
\usepackage[colorlinks]{hyperref}%%
%% ================================
%% ================================

%% ================================
%% Page size/borders config =======
\setlength{\oddsidemargin}{0in}  %%
\setlength{\evensidemargin}{0in} %%
\setlength{\marginparwidth}{0in} %%
\setlength{\marginparsep}{0in}   %%
\setlength{\voffset}{-0.5in}     %%
\setlength{\hoffset}{0in}        %%
\setlength{\topmargin}{0in}      %%
\setlength{\headheight}{54pt}    %%
\setlength{\headsep}{1em}        %%
\setlength{\textheight}{8.5in}   %%
\setlength{\footskip}{0.5in}     %%
%% ================================
%% ================================

%% =============================================================
%% Headers setup, environments, colors, etc.
%%
%% Header ------------------------------------------------------
\fancypagestyle{firstpage}
{
  \fancyhf{}
  \lhead{\includegraphics[height=4.5em]{LogoDFI.jpg}}
  \rhead{FI3104-1 \semestre\\
         Métodos Numéricos para la Ciencia e Ingeniería\\
         Prof.: \profesor}
  \fancyfoot[C]{\thepage}
}

\pagestyle{plain}
\fancyhf{}
\fancyfoot[C]{\thepage}
%% -------------------------------------------------------------
%% Environments -------------------------------------------------
\newmdenv[
  linecolor=gray,
  fontcolor=gray,
  linewidth=0.2em,
  topline=false,
  bottomline=false,
  rightline=false,
  skipabove=\topsep
  skipbelow=\topsep,
]{ayuda}
%% -------------------------------------------------------------
%% Colors ------------------------------------------------------
\definecolor{gray}{rgb}{0.5, 0.5, 0.5}
%% -------------------------------------------------------------
%% Aliases ------------------------------------------------------
\newcommand{\scipy}{\texttt{scipy}}
%% -------------------------------------------------------------
%% =============================================================
%% =============================================================================
%% CONFIGURACION DEL DOCUMENTO =================================================
%% Llenar con la información pertinente al curso y la tarea
%%
\newcommand{\tareanro}{3}
\newcommand{\fechaentrega}{29/09/2016 23:59 hrs}
\newcommand{\semestre}{2016B}
\newcommand{\profesor}{Valentino González}
%% =============================================================================
%% =============================================================================


\begin{document}
\thispagestyle{firstpage}

\begin{center}
  {\uppercase{\LARGE \bf Tarea \tareanro}}\\
  Fecha de entrega: \fechaentrega
\end{center}


%% =============================================================================
%% ENUNCIADO ===================================================================
\noindent{\large \bf Problema 1}

El objetivo de este problema es investigar cómo se comporta la interpolación
con polinomios versus la interpolación spline en algunos casos interesantes.

Considere la siguiente función Gaussiana:

$$ f(x) = e^{-x^2/0.05} $$

\noindent en el intervalo $[-1, 1]$. Divida el intervalo en 4 tramos
equiespaciados (es decir, samplee 5 puntos en el intervalo $[-1, 1]$). Ahora
interpole un polinomio (usando, por ejemplo, el método de Lagrange) a través de
esos 5 puntos. Haga lo mismo usando una interpolación spline.

Ahora aumente secuencialmente el número de puntos y compruebe cómo se comportan
los dos métodos (mejoran?, empeoran?, es lo que esperaba?).

\begin{ayuda}
  \noindent {\bf Nota.}

  Puede programar su propio método de Lagrange y/o spline, o puede utilizar
  alguna librería que le parezca adecuada. Investigue, por ejemplo, el módulo
  de interpolación de \texttt{scipy}. Si decide usar una librería, asegúrese de
  estar usando Lagrange para los polinomios. También asegúrese de comprender
  los detalles de la implementación spline que esté utilizando (qué pasa en los
  extremos, por ejemplo?). Incluya esta información en el informe.
\end{ayuda}

\vspace{1em}
\noindent{\large \bf Problema 2}

Una de las múltiples aplicaciones de los métodos de interpolación es la
reconstrucción de imágenes dañadas (por fallas en el detector u otros motivos).
Para esto, se requiere métodos tipo spline que funcionen en 2D.

Considere la siguiente imagen (archivo adjunto \texttt{galaxia.bmp}) que
contiene pixeles dañados (el archivo \texttt{mask.bmp} le indica cuales son los
pixeles dañados) y reconstrúyala. Para eso, reemplace los pixeles dañados por
el valor obtenido de la interpolación 2D basada en un grupo de pixeles
adyacentes no dañados.

\begin{ayuda}
  \small
  \noindent{\bf Ayuda.}

  El módulo \texttt{scikit-image} le permite leer imágenes en formatos estándar
  (por ejemplo, \texttt{jpg, bmp}). Las imágenes a color son leídas en 3
  arreglos 2D de tipo \texttt{numpy}, correspondientes a los colores rojo,
  verde y azul. El archivo \texttt{read\_image.py} contiene algunas
  instrucciones que le pueden ser útiles.

  No es necesario que escriba su propio algoritmo spline2D pero es importante
  que explique en detalle que hace el algoritmo que Ud. elija usar. Por
  ejemplo, \texttt{scipy.interpolate.SmoothBivariateSpline} es un wrapper de
  \texttt{FITPACK}. Si lo usa, explique qué es \texttt{FITPACK}, qué hace la
  función que está usando y qué parámetros utilizó al llamar a la función y por
  qué.

  Para hacer esta parte de la tarea de manera efectiva es útil entender el
  concepto de \emph{array slicing} en python y el uso de la función
  \texttt{numpy.where}. Estos conceptos no son exclusivos de python, sólo
  varían en su implementación en otros lenguages (como \texttt{matlab}).
\end{ayuda}

\pagebreak
\vspace{1em}
\noindent{\bf Instrucciones importantes.}
\begin{itemize}

  \item Utilice \texttt{git} durante el desarrollo de la tarea para mantener un
    historial de los cambios realizados. La siguiente
    \href{https://education.github.com/git-cheat-sheet-education.pdf}{cheat
      sheet} le puede ser útil. {\bf Esta vez revisaremos el uso apropiado de
    la herramienta y asignaremos una fracción del puntaje a este ítem.} Realice
    cambios pequeños y guarde su progreso (a través de \emph{commits})
    regularmente. No guarde código que no corre o compila (si lo hace por algún
    motivo deje un mensaje claro que lo indique). Escriba mensajes claros que
    permitan hacerse una idea de lo que se agregó de un \texttt{commit} al
    siguiente.

  \item También comenzaremos a revisar su uso correcto de python. Si define una
    función relativametne larga o con muchos parámetros, recuerde escribir el
    \emph{docstring} que describa los parámetros que recibe la función, el
    output, y el detalle de qué es lo que hace la función. Recuerde que
    generalmente es mejor usar varias funciones cortas (que hagan una sola cosa
    bien) que una muy larga (que lo haga todo).  Utilice nombres explicativos
    tanto para las funciones como para las variables de su código. El mejor
    nombre es aquel que permite entender qué hace la función sin tener que leer
    su implementación.

  \item La tarea se entrega subiendo su trabajo a github. Clone este
    repositorio (el que está en su propia cuenta privada), trabaje en el código
    y en el informe y cuando haya terminado asegúrese de hacer un último
    \texttt{commit} y luego un \texttt{push} para subir todo su trabajo a
    github.

  \item El informe debe ser entregado en formato \texttt{pdf}, este debe ser
    claro sin información de más ni de menos. Esto es importante, no escriba de
    más, esto no mejorará su nota sino que al contrario. Asegúrese de utilizar
    figuras efectivas y tablas para resumir sus resultados. Revise su
    ortografía.

\end{itemize}


%% FIN ENUNCIADO ===============================================================
%% =============================================================================

\end{document}
